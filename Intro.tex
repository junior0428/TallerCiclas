%preámbulo, esto no se vera en el docuemnto final. es como configuraciones
\documentclass[12pt, a4paper]{article}
\usepackage[utf8]{inputenc}%Para las acentuaciones de las informaciones o textos que ejecutaras %codificacion en teclado
\usepackage{lipsum}
\usepackage[spanish]{babel}
\usepackage{amsmath} %modo matematico 
\usepackage{amssymb}%para realizar simbolos
\usepackage{amsfonts}% escribir muchos fuentes en modo matematico
\usepackage{graphicx} %para insertar imagenes del exterior
\usepackage{xcolor} %colores 
\usepackage[left = 2.54cm, right = 2.54cm, top = 2.54cm, bottom = 2.54cm]{geometry}

\title{Primeros paso en latex}
\author{juniorantoniocalvomontanez }
\date{October 2021}

\begin{document}

\maketitle %para poner el preambulo

\section{Introducción}
        \noindent
    {Hola me llamo junior}\\
    \textit{Hola me llamo junior}\\
    \textsl{Hola me llamo junior}\\
    \textsc{Hola me llamo Junior}\\
    \emph{Hola me llamo junior}\\
    
    \noindent
    {\scriptsize Hola me llamo junior}\\
    {\footnotesize Hola me llamo junior}\\
    {\small Hola me llamo junior}\\
    
    {Hola me llamo junior}\\
    {\large hola me llamo junior}\\
    {\Large Hola me llamo junior}\\
    {\LARGE Hola me llamo junior}\\
\section{Objetivo}
\lipsum[7]
\end{document}
